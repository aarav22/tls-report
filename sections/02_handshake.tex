Cryptographic applications commonly require both confidentiality and message authentication. Confidentiality ensures that data is available only to those authorized to obtain it; usually it is realized through encryption.  Message authentication is the service that ensures that data has not been altered or forged by unauthorized entities; it can be achieved by using a Message Authentication Code (MAC). This service is also called data integrity. \textbf{Authenticated Encryption (AE)} \cite{ae} schemes ensure both data secrecy (confidentiality) and data integrity. These schemes can be constructed with either a \textbf{generic composition} that combines a CPA-secure cipher with a secure MAC, or to build them directly from a block cipher or a PRF without first constructing either a standalone cipher or MAC \cite{gradcourse}. The latter schemes are called \textbf{integrated schemes}. 

Let $(E, D)$ be a cipher and $(S, V)$ be a MAC. Let $k_{enc}$ be a cipher key and $k_{mac}$ be a MAC key. In a generic composition, these two primitives can be combined using two commonly used options: \textbf{Encrypt-then-MAC} and \textbf{MAC-then-Encrypt}. In the first option, the plaintext is encrypted using the cipher and the resulting ciphertext is authenticated using the MAC. In the second option, the plaintext is authenticated using the MAC and the resulting MAC tag is encrypted using the cipher (see Figure \ref{fig:encrypt-then-mac} and Figure \ref{fig:mac-then-encrypt}). Only the first method is secure for every combination of CPA-secure cipher and secure MAC. The intuition is that the MAC on the ciphertext prevents any tampering with the ciphertext. The second method is known to have attacks in some cases \cite{gradcourse}. 

We will now define the encryption scheme used in TLS. TLS uses \textbf{Authenticated Encryption with Associated Data (AEAD)} \cite{aead} to encrypt and authenticate data. AEAD schemes are a special case of AE schemes that allow the sender to authenticate additional data along with the message. This additional data is called the \textbf{associated data} (AD). The associated data is not encrypted but is authenticated along with the message.

We will now use the notation $\varepsilon = (E_s , D_s)$ to denote an authenticated encryption scheme.  

\begin{figure}[ht]
  \centering
  \begin{subfigure}[b]{0.4\textwidth}
    \centering
    \begin{tikzpicture}[node distance=0 cm,outer sep = 0pt,inner sep = 2pt]
      \tikzset{field/.style={align=center,shape=rectangle,minimum height=0.5cm,minimum width=17mm,draw}}
      \tikzset{largefield/.style={align=center,shape=rectangle,minimum height=0.5cm,minimum width=34mm,draw}}
    
        \node [largefield] (message) {\textit{m}};
        \node [largefield, below=of message] (empty) { };
        \node [largefield,below=of empty,fill=gray!30] (enc) {$c \leftarrow E(k_{enc}, m)$};
        \node [largefield,below=of enc] (mac) {$\text{tag } \leftarrow S(K_{mac}, c)$};
        \node [field,below=of mac, xshift=-8.5mm, fill=gray!30] (cipher) {$c$};
        \node [field,below=of mac, xshift=8.5mm] (tag) {tag};
    \end{tikzpicture}
    \caption{Encrypt-then-MAC}
    \label{fig:encrypt-then-mac}
  \end{subfigure}
  \begin{subfigure}[b]{0.4\textwidth}
    \centering
    \begin{tikzpicture}[node distance=0 cm,outer sep = 0pt,inner sep = 2pt]
        \tikzset{field/.style={align=center,shape=rectangle,minimum height=0.5cm,minimum width=17mm,draw}}
        \tikzset{largefield/.style={align=center,shape=rectangle,minimum height=0.5cm,minimum width=34mm,draw}}
    
        \node [largefield] (message) {\textit{m}};
        \node [largefield,below=of message] (mac) {$\text{tag } \leftarrow S(K_{mac}, m)$};
        \node [field,below=of mac, xshift=-8.5mm] (pkt-m) {$m$};
        \node [field,below=of mac, xshift=8.5mm] (tag) {tag};
        \node [largefield, below=of pkt-m, xshift=8.5mm] (empty) { };
        \node [largefield,below=of empty,fill=gray!30] (enc) {$c \leftarrow E(k_{enc}, (m, tag))$};
  
    \end{tikzpicture}
    \caption{MAC-then-Encrypt}
    \label{fig:mac-then-encrypt}
  \end{subfigure}
  \label{fig:encrypt-then-mac-mac-then-encrypt}
  \caption{Two different ways to combine a cipher and a MAC.}
  \hrulefill
\end{figure}

For consistency with the notation, we let P play the role of the client and Q play the role of the server. P and Q wish to setup a secure session. 

\textbf{TLS 1.3} supports both one-sided and mutual authentication. In most cases, authentication for the client is optional. In the notation, $(E_s , D_s)$ is a symmetric encryption scheme that provides authenticated encryption, such as AES-128 in GCM mode.  Algorithm $S$ refers to a MAC signing algorithm, such as HMAC-SHA256. Algorithms $Sig_P(\cdot )$ and $Sig_Q(\cdot )$ sign the provided data using P's or Q's signing keys. Finally, the hash functions $H_1, H_2$ are used to derive symmetric keys. They are built from HKDF (Apppendix I) with hash functions such as SHA-256. The notations are taken from \cite{gradcourse}.

The cipher-suites which determine the symmetric encryption scheme, the hash function in the MAC signing algorithm and HKDF are negotiated during the handshake. The cipher-suites are defined in \cite{rfc8446}. They are specified in Table \ref{tab:ciphersuites}. 

% TLS uses \textbf{Authenticated encryption} \cite{ae} that is a form of encryption that, in addition to providing confidentiality for the plaintext that is encrypted, provides a way to check its integrity and authenticity. \textbf{Authenticated Encryption with Associated Data (AEAD)} \cite{aead}, adds the ability to check the integrity and authenticity of some Associated Data (AD), also called "additional authenticated data", that is not encrypted.

% Many applications use an encryption method and a MAC together to provide both of those security services, with each algorithm using an independent key. More recently, the idea of providing both security services using a single cryptoalgorithm has become accepted \cite{rfc5116}. In this concept, the cipher and MAC are replaced by an Authenticated Encryption with Associated Data (AEAD) algorithm \cite{aead}.




\begin{table}[ht]
  \centering
  \begin{tabular}{|l|l|}
    \hline Description & Value \\
    \hline 
    TLS\_AES\_128\_GCM\_SHA256 & $\{0\times13,0\times01\}$ \\
    \hline TLS\_AES\_256\_GCM\_SHA384 & $\{0\times13,0\times02\}$ \\
    \hline TLS\_CHACHA20\_POLY1305\_SHA256 & $\{0\times13,0\times03\}$ \\
    \hline TLS\_AES\_128\_CCM\_SHA256 & $\{0\times13,0\times04\}$ \\
    \hline TLS\_AES\_128\_CCM\_8\_SHA256 & $\{0\times13,0\times05\}$ \\
    \hline
  \end{tabular}
  \caption{Cipher-Suites \cite{rfc8446}. Cipher suite names follow the naming convention: CipherSuite TLS\_AEAD\_HASH = VALUE;}
  \label{tab:ciphersuites}
\end{table}



\textbf{The protocol.} The handshake protocol 
% \begin{align*}
%     & \textbf { ClientHello } \\
%     & u:=g^\alpha, \mathcal{N}_{\mathrm{c}}, offer \\
%     & \textbf { ServerHello } \\
%     & v:=g^\beta, \mathcal{N}_{\mathrm{s}}, \text { mode } \\ 
%     & c_1:=E_{\mathrm{s}}\left(k_{\mathrm{sh}}, \text { CertReqest }\right) \\ 
%     & c_2:=E_{\mathrm{s}}\left(k_{\mathrm{sh}}, \operatorname{Cert}_Q\right) \\ 
%     & c_3:=E_{\mathrm{s}}\left(k_{\mathrm{sh}}, \text { Sig }\right. \\ 
%     & \left.c_4\left(u, \mathcal{N}_{\mathrm{c}}, \text { offer }, v, \mathfrak{N}_{\mathrm{s}}, \text { mode }, c_1, c_2\right)\right) \\ 
%     & c_4=E_{\mathrm{s}}\left(k_{\mathrm{sh}}, S\left(k_{\mathrm{sm}},\left(u, \mathfrak{N}_{\mathrm{c}}, \text { offer }, v, \mathfrak{N}_{\mathrm{s}}, \text { mode }, c_1, c_2, c_3\right)\right)\right) \\
%     & \left(k_{\mathrm{sh}}, k_{\mathrm{sm}}, k_{\mathrm{ch}}, k_{\mathrm{cm}}\right):=H_1\left(g^{\alpha \beta}, u, \mathcal{N}_{\mathrm{c}}, \text { offer }, v, \mathcal{N}_{\mathrm{s}}, \text { mode }\right) \\
%     & \left(k_{\mathrm{c} \rightarrow \mathrm{s}}, k_{\mathrm{s} \rightarrow \mathrm{c}}\right):=H_2\left(g^{\alpha \beta}, u, \mathcal{N}_{\mathrm{c}}, \text { offer, } v, \mathfrak{N}_{\mathrm{s}}, \text { mode }, c_1, \ldots, c_4\right) \\
%     & \text { ClientKeyExchange } \\
% \end{align*}

% \begin{figure}[ht]
%     \centering
%     \begin{tikzpicture}[block/.style={draw,minimum width=2cm,minimum height=1cm}, font=\sffamily] 
%     \node[block](C) {Client}; 
%     \node[block,right=9cm of C](S) {Server}; 
%     \draw[-latex] (C.15) -- (S.165)
%     node[midway,above]{Request(GET,POST,PUT,DELETE,HEAD,OPTION)}; 
%     \draw[-latex] (S.195) -- (C.-15) node[midway,above]{
%       $\begin{aligned}
%         & \textbf { ServerHello } \\
%         & v:=g^\beta, \mathcal{N}_{\mathrm{s}}, \text { mode } \\ 
%         & c_1:=E_{\mathrm{s}}\left(k_{\mathrm{sh}}, \text { CertReqest }\right) \\ 
%         & c_2:=E_{\mathrm{s}}\left(k_{\mathrm{sh}}, \operatorname{Cert}_Q\right) \\ 
%         & c_3:=E_{\mathrm{s}}\left(k_{\mathrm{sh}}, \text { Sig }\right. \\ 
%         & \left.c_4\left(u, \mathcal{N}_{\mathrm{c}}, \text { offer }, v, \mathfrak{N}_{\mathrm{s}}, \text { mode }, c_1, c_2\right)\right) \\ 
%         & c_4=E_{\mathrm{s}}\left(k_{\mathrm{sh}}, S\left(k_{\mathrm{sm}},\left(u, \mathfrak{N}_{\mathrm{c}}, \text { offer }, v, \mathfrak{N}_{\mathrm{s}}, \text { mode }, c_1, c_2, c_3\right)\right)\right) \\
%         & \left(k_{\mathrm{sh}}, k_{\mathrm{sm}}, k_{\mathrm{ch}}, k_{\mathrm{cm}}\right):=H_1\left(g^{\alpha \beta}, u, \mathcal{N}_{\mathrm{c}}, \text { offer }, v, \mathcal{N}_{\mathrm{s}}, \text { mode }\right) \\
%         & \left(k_{\mathrm{c} \rightarrow \mathrm{s}}, k_{\mathrm{s} \rightarrow \mathrm{c}}\right):=H_2\left(g^{\alpha \beta}, u, \mathcal{N}_{\mathrm{c}}, \text { offer, } v, \mathfrak{N}_{\mathrm{s}}, \text { mode }, c_1, \ldots, c_4\right) \\
%       \end{aligned}$
%     }; 
%     \end{tikzpicture}
%     \caption{TLS Handshake Sequence}
%     \label{fig:handshake}
%   \end{figure}